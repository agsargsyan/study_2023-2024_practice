\chapter*{Приложения}
\addcontentsline{toc}{chapter}{Приложения}

Ссылка на репозиторий: 
\url{https://github.com/agsargsyan/study_2022-2023_practice}


\section*{Программа симуляции в NS-2}

\begin{minted}[linenos,tabsize=2,breaklines]{bash}
- main.tcl
```
#Создать новый экземпляр объекта Symulator
set ns [new Simulator]

#Открыть трейс файл для nam, файл слишком большой, так что временно закомментируем
set nf [open output/out.nam w]
$ns namtrace-all $nf

#количество источников 
set N 20

#создание узлов
source "nodes.tcl"

#очередь		
source "queue.tcl"

#настройка времени моделирования  		
source "timing.tcl" 		

#визуализация
source "nam.tcl"   		

#процедура finish
source "finish.tcl"                                                                         

#Запуск программы
$ns run

 
```
- nodes.tcl
```
set node_(r0) [$ns node]  #первый маршрутизатор
set node_(r1) [$ns node]  #второй маршрутизатор	

for {set i 0} {$i < $N} {incr i} {
	set node_(s$i) [$ns node] 		#источник
	set node_(s[expr $N + $i]) [$ns node]	#приемник
	}

#линки между маршрутизаторами и другими узлами(размер буфера, время, тип очереди)
for {set i 0} {$i < $N} {incr i} {
	$ns duplex-link $node_(s$i) $node_(r0) 100Mb 20ms DropTail
	$ns duplex-link $node_(s[expr $N + $i]) $node_(r1) 100Mb 20ms DropTail
}

#линки между маршрутизаторами(размер буфера, время, тип очереди)
$ns simplex-link $node_(r0) $node_(r1) 20Mb 15ms RED
$ns simplex-link $node_(r1) $node_(r0) 15Mb 20ms DropTail

# Агенты и приложения:
for {set t 0} {$t < $N} {incr t} {
	$ns color $t green
	set tcp($t) [$ns create-connection TCP/Reno $node_(s$t) TCPSink $node_(s[expr $N + $t]) $t]
	$tcp($t) set window_ 32
	$tcp($t) set maxcwnd_ 32
	set ftp($t) [$tcp($t) attach-source FTP]
}

```

- nam.tcl
```
#визуализация цветов, формы, располажения узлов в nam
$node_(r0) color "red"
$node_(r1) color "red"
$node_(r0) label "RED"
$node_(r1) shape "square"
$node_(r0) label "square"

$ns simplex-link-op $node_(r0) $node_(r1) orient right
$ns simplex-link-op $node_(r1) $node_(r0) orient left
$ns simplex-link-op $node_(r0) $node_(r1) queuePos 0
$ns simplex-link-op $node_(r1) $node_(r0) queuePos 0

for {set m 0} {$m < $N} {incr m} {
	$ns duplex-link-op $node_(s$m) $node_(r0) orient right
	$ns duplex-link-op $node_(s[expr $N + $m]) $node_(r1) orient left 
}

for {set i 0} {$i < $N} {incr i} {
	$node_(s$i) color "blue"
	$node_(s$i) label "ftp"

}
```

- queue.tcl
```
#Лимит очереди
$ns queue-limit $node_(r0) $node_(r1) 300
$ns queue-limit $node_(r1) $node_(r0) 300


# Мониторинг размера окна TCP
set windowVsTime [open output/WvsT w]
set qmon [$ns monitor-queue $node_(r0) $node_(r1) [open output/qm.out w]]
[$ns link $node_(r0) $node_(r1)] queue-sample-timeout


# Формирование файла с данными о размере окна TCP
proc plotWindow {tcpSource file} {
   global ns
   set time 0.01
   set now [$ns now]
   set cwnd [$tcpSource set cwnd_]
   puts $file "$now $cwnd"
   $ns at [expr $now+$time] "plotWindow $tcpSource $file"
}

# Мониторинг очереди:
set redq [[$ns link $node_(r0) $node_(r1)] queue]
$redq set thresh_ 75
$redq set maxthresh_ 150
$redq set q_weight_ 0.002
$redq set linterm_ 10
$redq set drop-tail_ true

$redq set queue-in-bytes false
set tchan_ [open output/all.q w]
$redq trace curq_
$redq trace ave_
$redq attach $tchan_

#Для реализации разных модификаций RED,
$redq set gentle_ false 

#$redq set nonlinear_ 1
#$redq set hyperbola_ 1 
#$redq set quadratic_linear_ 1
#$redq set three_sections_ 1
#$redq set exponential_ 1
#$redq set smart_ 1
#$redq set double_slope_ 1

# Группа адаптивных алгоритмов
#$redq set adaptive_ 1
#$redq set feng_adaptive_ 1
#$redq set refined_adaptive_ 1
#$redq set fast_adaptive_ 1
#$redq set powared_ 1

```

- timing.tcl
```
#Задаем время симуляции
for {set r 0} {$r < $N} {incr r} {
        $ns at 0.0 "$ftp($r) start" 
        $ns at 1.0 "plotWindow $tcp($r) $windowVsTime"
        $ns at 24.0 "$ftp($r) stop"
}

$ns at 25.0 "finish"
```

- finish.tcl
```
#Finish procedure
proc finish {} {
   global ns nf
   $ns flush-trace
   close $nf
   global tchan_
   set awkCode {  
      {#запись данных в файлы очереди и средней очереди
         if ($1 == "Q" && NF>2) {
            print $2, $3 >> "output/temp.q";
            set end $2
         }
         else if ($1 == "a" && NF>2)
         print $2, $3 >> "output/temp.a";
      }
   }

   set f [open output/temp.queue w]
   puts $f "TitleText: RED"
   puts $f "Device: Postscript"

   if { [info exists tchan_] } {
      close $tchan_
   }
   #обновление данных
   exec rm -f output/temp.q output/temp.a
   exec touch output/temp.a output/temp.q

   exec awk $awkCode output/all.q

   puts $f \"queue
   exec cat output/temp.q >@ $f
   puts $f \n\"ave_queue
   exec cat output/temp.a >@ $f
   close $f
   # вывод в xgraph
   exec xgraph -bb -tk -x time -t "TCPRenoCWND" output/WvsT &
   exec xgraph -bb -tk -x time -y queue output/temp.queue &
   exit 0
}
\end{minted}
% \end{verbatim}


\section*{red.cc adaptive REDs}

\begin{minted}[linenos,tabsize=2,breaklines]{C++}

void REDQueue::updateMaxP(double new_ave, double now)
{
	double part = 0.4*(edp_.th_max - edp_.th_min);
	// AIMD rule to keep target Q~1/2(th_min+th_max)
	if ( new_ave < edp_.th_min + part && edv_.cur_max_p > edp_.bottom) {
		// we increase the average queue size, so decrease max_p
		edv_.cur_max_p = edv_.cur_max_p * edp_.beta;
		edv_.lastset = now;
	} else if (new_ave > edp_.th_max - part && edp_.top > edv_.cur_max_p ) {
		// we decrease the average queue size, so increase max_p
		double alpha = edp_.alpha;
                        if ( alpha > 0.25*edv_.cur_max_p )
			alpha = 0.25*edv_.cur_max_p;
		edv_.cur_max_p = edv_.cur_max_p + alpha;
		edv_.lastset = now;
	} 
}

void REDQueue::updateMaxP_refined_adaptive(double new_ave, double now)
{
  	double part = 0.48*(edp_.th_max - edp_.th_min);
 	if ( new_ave < edp_.th_min + part && edv_.cur_max_p > edp_.bottom) {
 		edv_.cur_max_p = edv_.cur_max_p * (1.0 - (0.17 * ((edp_.th_min + part) - new_ave) / ((edp_.th_min + part) - edp_.th_min))); 
 		edv_.lastset = now;
 		double maxp = edv_.cur_max_p;
 	} else if (new_ave > edp_.th_max - part && edp_.top > edv_.cur_max_p ) {
 		double alpha = edp_.alpha;
 		alpha = 0.25 * edv_.cur_max_p * ((new_ave - (edp_.th_max - part)) / (edp_.th_max - part));
 		edv_.cur_max_p = edv_.cur_max_p + alpha;
 		edv_.lastset = now;
 		double maxp = edv_.cur_max_p;
 	}
}


void REDQueue::updateMaxP_powared(double new_ave, double now)
{
  	double target = 0.5*(edp_.th_max + edp_.th_min);
  	int k = edp_.pwk;
  	int b = edp_.pwb;
  	int r = edp_.bf_size;
  	double v_ave = edv_.v_ave;
  	
  	
  	
  	double delta1 = abs(pow((v_ave - target)/(b * target), k));
  	double delta2 = abs(pow((target - v_ave)/(b * (r -target)), k));
 	
 	if ( new_ave < target && edv_.cur_max_p > edp_.bottom) {
 		edv_.cur_max_p = edv_.cur_max_p - delta1; 
 		edv_.lastset = now;
 		double maxp = edv_.cur_max_p; 
 	} else if (new_ave > target && edp_.top > edv_.cur_max_p ) {
 		edv_.cur_max_p = edv_.cur_max_p + delta2;
 		edv_.lastset = now;
 		double maxp = edv_.cur_max_p;
 	}
}

void REDQueue::updateMaxP_fast_adaptive(double new_ave, double now){
	  	double part = 0.48*(edp_.th_max - edp_.th_min);
		if ( new_ave < edp_.th_min + part && edv_.cur_max_p > edp_.bottom) {
 		edv_.cur_max_p = edv_.cur_max_p * (1.0 - (0.0385 * ((edp_.th_min + part) - new_ave) / ((edp_.th_min + part) - edp_.th_min))); 
 		edv_.lastset = now;
 		double maxp = edv_.cur_max_p;
 	} else if (new_ave > edp_.th_max - part && edp_.top > edv_.cur_max_p ) {
 		double alpha = edp_.alpha;
 		alpha = 0.0412 * edv_.cur_max_p * (new_ave - part) / part;
 		edv_.cur_max_p = edv_.cur_max_p + alpha;
 		edv_.lastset = now;
 		double maxp = edv_.cur_max_p;
 	}


}

\end{minted}

\section*{red.cc calculate drop probability}

\begin{minted}[linenos,tabsize=2,breaklines]{C++}
/*
 * Calculate the drop probability.
 */
double
REDQueue::calculate_p_new(double v_ave, double th_max, int gentle, double v_a, 
	double v_b, double v_c, double v_d, double max_p)
{	
	double target;
	double exponenta = 2.7182818285;
	double th_min = edp_.th_min;
	double p;
	if (gentle && v_ave >= th_max) {
		// p ranges from max_p to 1 as the average queue
		// size ranges from th_max to twice th_max 
		p = v_c * v_ave + v_d;
        } else if (!gentle && v_ave >= th_max) { 
                // OLD: p continues to range linearly above max_p as
                // the average queue size ranges above th_max.
                // NEW: p is set to 1.0 
                p = 1.0;
        } else if (edp_.quadratic_linear == 1) {
        	target = 2 * ((th_min + th_max)/3) - th_min;
        	if(v_ave < target){
        		p = 9 * max_p * ((v_ave-th_min)/(2*(th_max-2*th_min))) * ((v_ave-th_min)/(2*(th_max-2*th_min)));
        	} else if (v_ave >= target) {
        		p = max_p + 3*(1-max_p)*((v_ave-target)/(th_max+th_min));
        	}
        } else if (edp_.improved == 1) {
        	target = ((th_min + th_max)/3) + th_min;
        	if(v_ave < target){
        		p = 9 * max_p * ((v_ave-th_min)/(th_max + th_min)) * ((v_ave-th_min)/(th_max + th_min));
        	} else if (v_ave >= target) {
        		p = max_p + 3*(1-max_p)*(v_ave-target)/(2*(th_max - 2 * th_min));
        	}
        } else if (edp_.smart == 1) {
        	target = ((th_max - th_min)/2) + th_min;
        	if(v_ave < target){
        		p = max_p * pow(((v_ave-th_min)/(th_max - th_min)), 2);
        	} else if (v_ave >= target) {
        		p = max_p * pow(((v_ave-th_min)/(th_max - th_min)), 0.5);
        	}
        } else if (edp_.three_sections == 1){
        	double delta = (th_min+th_max/3);
        	if (v_ave < (th_min + delta)){
        		p = 9 * max_p * pow((v_ave-th_min)/(th_max-th_min), 3) ;
        		}
        	else if ((v_ave >= th_min + delta) && (v_ave < th_min + 2 * delta)){
        		p = max_p * (v_ave-th_min)/(th_max-th_min);
        		}
        	else if (v_ave >= th_min + 2* delta){
        		p =  9 * max_p * pow((v_ave-th_min)/(th_max-th_min), 3) + max_p;
        	} 
        }
        else if (edp_.double_slope == 1) {
        	double a = (2-2* edp_.omega)/(th_max - th_min);
        	double b = (2 * edp_.omega)/(th_max - th_min);
        	target = ((th_max + th_min)/2);
        	if(v_ave < target){
        		p = a * (v_ave-th_min);
        	} else if (v_ave >= target) {
        		p = 1 - edp_.omega + b * (v_ave - target);
        	}
        }
        else {
                // p ranges from 0 to max_p as the average queue
                // size ranges from th_min to th_max 
                p = v_a * v_ave + v_b;
                // p = (v_ave - th_min) / (th_max - th_min)
                
                /* Added by Mohit P. Tahiliani for Nonlinear RED (NLRED) - Start */
				if(edp_.nonlinear == 1){
					p *= p;		// This ensures probability is a quadratic function of "average queue size" as specified in NLRED Paper
				}
				else if (edp_.hyperbola == 1){
					p *= 1/p;	// This ensures probability is a hyperbola function of "average queue size" as specified in HRED Paper
				}
				else if (edp_.exponential == 1){ // Used for RED_e
					p = (pow(exponenta, v_ave)-pow(exponenta, th_min))/(pow(exponenta, th_max)-pow(exponenta, th_min));
				}
                p *= max_p; 
        }
	if (p > 1.0)
		p = 1.0;
	return p;
}
\end{minted}

\section*{red.h}

\begin{minted}[linenos,tabsize=2,breaklines]{C++}

	int feng_adaptive;	/* adaptive RED: Use the Feng et al. version */
	int refined_adaptive;	/* Added by Mohit P. Tahiliani for Refined Adaptive RED (Re-ARED) */
	int stabilized_adaptive;/* Added Stabillized Adaptive RED (SARED) */
	int nonlinear;		/* Added for Nonlinear RED (NLRED) */
	int hyperbola;		/* Added for Hyperbola RED (HRED)*/
	int quadratic_linear;   /* Added for Quadratic linear RED*/
	int three_sections;	/* Added for 3sections RED*/
	int exponential;	/* Added for exponential RED*/
	int improved;		/* Added for improved RED*/
	int smart;		/* Added for smart RED*/
	int modified;		/* Added for modified RED*/	 
\end{minted}

\section*{ns-default.tcl}

\begin{minted}[linenos,tabsize=2,breaklines]{C++}

/*Added for new RED alghorithms*/

Queue/RED set nonlinear_ 0
Queue/RED set hyperbola_ 0
Queue/RED set quadratic_linear_ 0
Queue/RED set three_sections_ 0
Queue/RED set exponential_ 0
Queue/RED set improved_ 0
Queue/RED set smart_ 0
Queue/RED set modified_ 0
\end{minted}


\section*{out.gp}

\begin{minted}[linenos,tabsize=2,breaklines]{C++}
#! /usr/bin/gnuplot -persist
set terminal postscript eps enhanced adobeglyphnames
set output "test.ps"
set encoding utf8

set xrange [0:26]

set terminal postscript eps
set output "av_queues_1GNl.eps"
set xlabel "Время [c]"
set ylabel "Длина очереди [пакеты]"
set title "Средневзвешенная очередь на линке при разных алгоритмах"
plot "classic.a" with lines linestyle 1 lt 1 lw 2 title "RED", "gentle.a" with lines linestyle 1 lt 2 lw 2 title "GRED", "nonlinear.a" with lines linestyle 1 lt 3 lw 2 title "NLRED" 

set terminal postscript eps
set output "av_queues_HTNl.eps"
set xlabel "Время [c]"
set ylabel "Длина очереди [пакеты]"
set title "Средневзвешенная очередь на линке при разных алгоритмах"
plot "hyperbola.a" with lines linestyle 1 lt 1 lw 2 title "HRED", "3.a" with lines linestyle 1 lt 2 lw 2 title "TRED", "nonlinear.a" with lines linestyle 1 lt 3 lw 2 title "NLRED" 
\end{minted}







\section*{Программа симуляции в mininet}

\begin{minted}[linenos,tabsize=2,breaklines]{bash}
-#!/usr/bin/env python

"""
Simple experiment.
Output: ping.dat
"""
from mininet.net import Mininet
from mininet.node import Controller
from mininet.cli import CLI

from mininet.log import setLogLevel, info
import time

def emptyNet():

   "Create an empty network and add nodes to it."

   net = Mininet( controller=Controller, waitConnected=True )

   info( '*** Adding controller\n' )
   net.addController( 'c0' )

   info( '*** Adding hosts\n' )
   h1 = net.addHost( 'h1', ip='10.0.0.1' )
   h2 = net.addHost( 'h2', ip='10.0.0.2' )

   info( '*** Adding switch\n' )
   s1 = net.addSwitch( 's1' )

   info( '*** Creating links\n' )
   net.addLink( h1, s1, bw = 10 )
   net.addLink( h2, s1, bw = 10 )

   info( '*** Starting network\n')
   net.start()

   info( '*** Set red options\n')
   s1.cmdPrint( 'tc qdisc add dev s1-eth1 root handle 1: red limit 1000000  max 30000 min 60000 burst 80 avpkt 1000 bandwidth 10Mbit ' )
   
   #для ARED
   #s1.cmdPrint( 'tc qdisc add dev s1-eth1 root handle 1: red limit 1000000  max 30000 min 60000 burst 80 avpkt 1000 bandwidth 10Mbit adaptive' )	


   info('*** Traffic generation\n')
   h2.cmdPrint('iperf3 -s -D -1 ')
   time.sleep(21) # Wait 21 seconds
   h1.cmdPrint('iperf3 -c', h2.IP(),'-J > iperf_result.json')

   info( '*** Ping\n')
   h1.cmdPrint( 'ping -c 10', h2.IP(), '| grep "time=" | awk \'{print $5, $7}\' | sed -e \'s/time=//g\' -e\'s/icmp_seq=//g\' > ping.dat' )

   info( '*** Stopping network' )
   net.stop()

if __name__ == '__main__':
    setLogLevel( 'info' )
    emptyNet()
    
% \end{verbatim}
\end{minted}


\section*{Makefile}

\begin{minted}[linenos,tabsize=2,breaklines]{bash}
all: ping.dat ping.pdf plot

ping.dat:
	sudo python lab_netem_i.py
	sudo chown mininet:mininet ping.dat

ping.pdf: ping.dat
	  ./ping_plot

plot: iperf_result.json
	plot_iperf.sh iperf_result.json

clean:
	-rm -f *.dat *.pdf *.json *.csv
	-rm -rf results
\end{minted}

\section*{}

\begin{minted}[linenos,tabsize=2,breaklines]{bash}

#!/usr/bin/gnuplot --persist

set terminal png crop
set output 'ping.pdf'
set xlabel "Sequence number"
set ylabel "Delay (ms)"
set grid
plot "ping.dat" with lines

\end{minted}







